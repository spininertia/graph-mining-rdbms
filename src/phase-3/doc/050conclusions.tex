% The proposed method {\em someMETHOD}
% has the following advantages:
% \bit
% \item it gives better classification accuracy than all 10 competitors we tried
% \item its accuracy is very close to the very best competitor
%       in the {\em UCR Insect Classification Contest}.
% \item it is scalable
% \eit

In this project, we investigate the major questions discussed in graph mining. We 
explored and solved the following problems:

\begin{itemize}
    \item We summarize the importance of graph mining techniques and propose our approach
    to this problem, which is Relational Database Management System.
    \item We conduct extensive survey about 7 graph mining tasks, each team member has read
    at least six papers each for the tasks. 
    \item For task 1, we calculate degree distribution of each node and do visualization. By 
    observing the plot, we conclude that {\bf social network} data exhibits {\bf power law}, while
    this is not a general rule, for instance, Roadnet dataset doest not show power law.
    \item For task 2, we calculate the pagerank of each node in a graph, namely the importance of each
    node. By visualize the pagerank distribution in a rank-frequency plot, we again observe {\bf power law}.
    \item For task 3, we compute the weakly connected components for a graph. By observing the plot, we find
    that there is a Giant Connected Component(GCC) which contains the majority of nodes in a graph. And the frequency-size 
    plot exhibits {\bf power law}.
    \item For task 4, we calculate the radius for every node in a graph using an approximate algorithm.
    By visualizing the result as radius plot, we find that most graphs has a {\bf single-modal} or {\bf bi-modal} radius distribution. 
    \item For task 5. We tackle the problem of calculating eigenvalue of an adjacent matrix by Lanczos mathod. We 
    build a Python matrix operation library which wraps low level linear algebra operations of SQL. 
    \item For task 6. We implement the belief propagation using FABP algorithm. By conducting experiments on large datasets,
    we successfully perform {\bf semi-supervised learning} using partially available labels to inference the labels of other nodes in the graph. 
    \item For task 7. We deal with the problem of count triangle(global or local) in a graph by calculating the
    eigenvalue of the adjacent matrix. Through the rank-frequency plot of local triangle distribution, we again
    observe {\bf power law}.
    \item We finished 2 innovative tasks, namely shortest path, minimum spanning tree. 
    \item We provide proof of correctness for each task we implemented. 
\end{itemize}
