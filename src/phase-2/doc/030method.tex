We implement all the graph mining algorithms using Postgres's embedded PL/pgSQL programming language, which supports many advanced features, like user defined function, aggregate, etc. It also has a sophisticated query execution engine, which we think is the most critical component of Postgres.

The following are the algorithms we plan to implement. The reason we choose them is that there are a lot of implementation in other platform like MapReduce, we can compare our SQL version with them to draw a clusion about SQL's unique charastic in solving data analytic tasks.

\begin{description}
  \item[PageRank:] Determine the importance of every node of a graph based on its connectivity. 
  \item[Connected Components:] Partition the nodes of a graph also based on its connectivity.
\end{description}

\subsection{Pagerank}
The algorithm of pagerank is Power method. We do matrix multiplication continusly until the change in pagerank is small. 
\begin{algorithm}
\caption{Pagerank}
\begin{algorithmic}
\STATE{Bulk load graph into an edge table in database.}
\STATE{Initialization(damper factor=0.85, max iteration = 100, epsilon = 0.0001)}
\STATE{Build a weight matrix trans, initialize pagerank p}
\REPEAT
\STATE{For each node i, update its pagerank with its old value and its income node's pagerank.}
\UNTIL{Convergence}
\end{algorithmic}
\end{algorithm}

\subsection{Weakly connected components}
In terms of the implementation of weakly connected components. We borrow the idea of HCC method from the "PEGASUS" paper.\cite{Kang09}
The algorithm can be described as following:
\begin{algorithm}
\caption{Weakly Connected Component}
\begin{algorithmic}
\STATE{Bulk load graph into an edge table in database.}
\STATE{Create a component table, where each entry contains a node id, and the component id.}
\STATE{Initialize the component table where component id equls node id.}
\REPEAT
\STATE{For each node, assign the minimum component id of its neighbors as the new componnet id of this node.}
\UNTIL{Convergence}
\end{algorithmic}
\end{algorithm}

After several rounds of iteration, the nodes in the same connected component will share the same component id.
The number of iterations for convergence can be proved to be upper bounded by the diameter of the graph.
