\subsection{Labor Division}

The team performed the following tasks
\bit
\item Implementation of Task 1,2 [Wei Chen]
\item Implementation of Task 3 [Siping Ji]
\item Remain tasks, scores in [2,4], the workload will divide by half, but detailed assignment is not determined. For harder task, we plan to adopt peer development, working together. 
\item Data collection [all]
\item Experiments on the real data [all]
\eit

\subsection{Project Development}
All the project related activity(code, report) are managed by Github\footnote{https://www.github.com}, a collaborative development community. All tasks are mapped into \emph{issues}, each phase is a \emph{milestone}. We fork from a central repository, when we finish our assigned tasks(issue), we send a pull request to the central repository. We are responsible for code review each other's code, then merge into the central repository. It's easy to see each team member's contribution by review the history of pull request and commit. The project address is here.\footnote{https://github.com/essex405/graph-mining-rdbms}

\subsection{Plan}

\begin{center}
Phase 1\\
\begin{tabular}{|c|c|c|}
\hline
Task & Due & Member \\\hline
Task1 & 10/07/13 & Wei\\\hline
Task2 & 10/07/13 & Wei\\\hline
Task3 & 10/07/13 & Siping\\\hline
\end{tabular}
\end{center}

\begin{center}
Phase 2\\
\begin{tabular}{|c|c|c|}
\hline
Task4 & 11/05/13 & Siping\\\hline
Task5 & 11/05/13 & Wei\\\hline
Task6 & 11/05/13 & Siping\\\hline
Task7 & 11/10/13 & Wei\\\hline
Task8 & 11/10/13 & Siping\\\hline
Final & 11/20/13 & Wei, Siping\\\hline
Poster & 11/20/13 & Wei, Siping\\\hline
\end{tabular}
\end{center}

\subsection{Additional Task}
In addition to the default tasks,  we plan to complete another two:  shortest path and minimum spanning tree. 

\subsection{List of Innovations}
\bit
\item We plan to extend the algorithms for weighted graphs; directed graphs; graphs with negative edge weights.
\item We will explore the best way to store edges in the relational database by using indexing.
\item We will do analysis about peak performance SQL implementation for each type of question, and what size of dataset can be efficiently analyzed in a single machine.
\eit

\subsection{Unit Tests}
The host language we use is Python, thus we plan to use its internal unit test framework\footnote{http://docs.python.org/2/library/unittest.html} as our testing module. We did unit tests for following modules:
\bit
\item Matrix Vector multiplication, abstract out the some basic matrix-vector, matrix-matrix operations. 
\item SQL user defined function bit-or
\item SQL user defined function fm-size
\eit






